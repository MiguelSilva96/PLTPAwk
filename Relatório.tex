%
% Layout retirado de http://www.di.uminho.pt/~prh/curplc09.html#notas
%
\documentclass{report}
\usepackage[portuges]{babel}
\usepackage[utf8]{inputenc}
%\usepackage[latin1]{inputenc}

\usepackage{url}
\usepackage{enumerate}

%\usepackage{alltt}
%\usepackage{fancyvrb}
\usepackage{listings}
\usepackage{eurosym}
%LISTING - GENERAL
\lstset{
    language=Awk,
    basicstyle=\ttfamily\small,
    numberstyle=\footnotesize,
    numbers=left,
    frame=single,
    tabsize=2,
    title=\lstname,
    escapeinside={\%*}{*)},
    breaklines=true,
    breakatwhitespace=true,
    framextopmargin=2pt,
    framexbottommargin=2pt,
    inputencoding=utf8,
    extendedchars=true,
    showspaces=false,
    showstringspaces=false,
    literate={á}{{\'a}}1 {é}{{\'e}}1 {í}{{\'i}}1 {ó}{{\'o}}1 {ú}{{\'u}}1
    {Á}{{\'A}}1 {É}{{\'E}}1 {Í}{{\'I}}1 {Ó}{{\'O}}1 {Ú}{{\'U}}1
    {à}{{\`a}}1 {è}{{\`e}}1 {ì}{{\`i}}1 {ò}{{\`o}}1 {ù}{{\`u}}1
    {À}{{\`A}}1 {È}{{\'E}}1 {Ì}{{\`I}}1 {Ò}{{\`O}}1 {Ù}{{\`U}}1
    {â}{{\^a}}1 {ê}{{\^e}}1 {î}{{\^i}}1 {ô}{{\^o}}1 {û}{{\^u}}1
    {ã}{{\~a}}1 {º}{{\textsuperscript{o}}}1 {ç}{{\c c}}1 {Ç}{{\c C}}1
    {€}{{\euro}}1
}


%
%\lstset{ %
%	language=Java,							% choose the language of the code
%	basicstyle=\ttfamily\footnotesize,		% the size of the fonts that are used for the code
%	keywordstyle=\bfseries,					% set the keyword style
%	%numbers=left,							% where to put the line-numbers
%	numberstyle=\scriptsize,				% the size of the fonts that are used for the line-numbers
%	stepnumber=2,							% the step between two line-numbers. If it's 1 each line
%											% will be numbered
%	numbersep=5pt,							% how far the line-numbers are from the code
%	backgroundcolor=\color{white},			% choose the background color. You must add \usepackage{color}
%	showspaces=false,						% show spaces adding particular underscores
%	showstringspaces=false,					% underline spaces within strings
%	showtabs=false,							% show tabs within strings adding particular underscores
%	frame=none,								% adds a frame around the code
%	%abovecaptionskip=-.8em,
%	%belowcaptionskip=.7em,
%	tabsize=2,								% sets default tabsize to 2 spaces
%	captionpos=b,							% sets the caption-position to bottom
%	breaklines=true,						% sets automatic line breaking
%	breakatwhitespace=false,				% sets if automatic breaks should only happen at whitespace
%	title=\lstname,							% show the filename of files included with \lstinputlisting;
%											% also try caption instead of title
%	escapeinside={\%*}{*)},					% if you want to add a comment within your code
%	morekeywords={*,...}					% if you want to add more keywords to the set
%}

\usepackage{xspace}

\parindent=0pt
\parskip=2pt

\setlength{\oddsidemargin}{-1cm}
\setlength{\textwidth}{18cm}
\setlength{\headsep}{-1cm}
\setlength{\textheight}{23cm}

\def\darius{\textsf{Darius}\xspace}
\def\antlr{\texttt{AnTLR}\xspace}
\def\pe{\emph{Publicação Eletrónica}\xspace}

\def\titulo#1{\section{#1}}
\def\super#1{{\em Supervisor: #1}\\ }
\def\area#1{{\em \'{A}rea: #1}\\[0.2cm]}
\def\resumo{\underline{Resumo}:\\ }


%%%%\input{LPgeneralDefintions}

\title{Programação Imperativa (?º ano de Curso)\\ \textbf{Trabalho Prático N}\\ Relatório de Desenvolvimento}
\author{Nome-Aluno1\\ (numero) \and Nome-Aluno2\\ (numero) }
\date{\today}

\begin{document}

\maketitle

\begin{abstract}
Isto é um resumo do relatório de \pe focando o contexto do trb (muito sucinto),
os objectivos concretos e os resultados atingidos.\\
Algum texto curto mas que entusiasme à leitura do relatório de \pe.
\end{abstract}

\tableofcontents


\chapter{Introdução} \label{intro}



\begin{description}
  \item [Enquadramento] \textbf{bla bla} bla bla
  \item [Conteúdo do documento] \textsf{ble ble} \texttt{ble} ble
  \item [Resultados -- pontos a evidenciar] \textit{bli bli bli bli}
  \item [Estrutura do documento] \underline{blo blo blo}
\end{description}

letras gregas são estas $ \alpha \beta \gamma \delta $ que aqui demonstro

exemplo simples de fração \[ \frac{\frac{a * b + c}{4-3}}{3*5} \] simples

Mais exemplos de listas enumeradas mas agora com letras:
\begin{enumerate}[a)]
\item Listar todas as Pessoas identificadas, sem repetições;
\item Listar os Países e Cidades marcadas;
\item Listar as Organizações.
\end{enumerate}

A mesma enumeração mas no standard numérico
\begin{description}
\item[Etape 1:] Listar todas as Pessoas identificadas, sem repetições;
\item[Etape 2:] Listar os Países e Cidades marcadas;
\item[Etape 3:] Listar as Organizações.
\end{description} 

\section*{Estrutura do Relatório} \
explicar como está organizado o documento, referindo os capítulos existentes em
e a sua articulação explicando o conteúdo de cada um.
No capítulo~\ref{ae} faz-se uma análise detalhada do problema proposto
de modo a poder-se especificar  as entradas, resultados e formas de transformação.\\
etc. \ldots\\
No capítulo~\ref{concl} termina-se o relatório com uma síntese do que foi dito,
as conclusões e o trabalho futuro

\chapter{Análise e Especificação} \label{ae}
\section{Descrição informal do problema}
\section{Especificação do Requisitos}
\subsection{Dados}
\subsection{Pedidos}
\subsection{Relações}

\chapter{Concepção/desenho da Resolução}
\section{Estruturas de Dados}
\section{Algoritmos}

\chapter{Codificação e Testes}
\section{Alternativas, Decisões e Problemas de Implementação}
\section{Testes realizados e Resultados}
Mostram-se a seguir alguns testes feitos (valores introduzidos) e
os respectivos resultados obtidos:



\chapter{Conclusão} \label{concl}
Síntese do Documento~.\\
Estado final do projecto; Análise crítica dos resultados~.\\
Trabalho futuro.

\appendix
\chapter{Código do Programa}

Lista-se a seguir o código  do programa  que foi desenvolvido.

\lstinputlisting{procViaverde}%input de um ficheiro

\bibliographystyle{alpha}
\bibliography{relprojLayout}



\end{document} 