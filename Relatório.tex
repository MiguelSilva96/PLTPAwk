%
% Layout retirado de http://www.di.uminho.pt/~prh/curplc09.html#notas
%
\documentclass{report}
\usepackage[portuges]{babel}
\usepackage[utf8]{inputenc}
%\usepackage[latin1]{inputenc}

\usepackage{url}
\usepackage{enumerate}

%\usepackage{alltt}
%\usepackage{fancyvrb}
\usepackage{listings}
\usepackage{eurosym}
%LISTING - GENERAL
\lstset{
    language=Awk,
    basicstyle=\ttfamily\small,
    numberstyle=\footnotesize,
    numbers=left,
    frame=single,
    tabsize=2,
    title=\lstname,
    escapeinside={\%*}{*)},
    breaklines=true,
    breakatwhitespace=true,
    framextopmargin=2pt,
    framexbottommargin=2pt,
    inputencoding=utf8,
    extendedchars=true,
    showspaces=false,
    showstringspaces=false,
    literate={á}{{\'a}}1 {é}{{\'e}}1 {í}{{\'i}}1 {ó}{{\'o}}1 {ú}{{\'u}}1
    {Á}{{\'A}}1 {É}{{\'E}}1 {Í}{{\'I}}1 {Ó}{{\'O}}1 {Ú}{{\'U}}1
    {à}{{\`a}}1 {è}{{\`e}}1 {ì}{{\`i}}1 {ò}{{\`o}}1 {ù}{{\`u}}1
    {À}{{\`A}}1 {È}{{\'E}}1 {Ì}{{\`I}}1 {Ò}{{\`O}}1 {Ù}{{\`U}}1
    {â}{{\^a}}1 {ê}{{\^e}}1 {î}{{\^i}}1 {ô}{{\^o}}1 {û}{{\^u}}1
    {ã}{{\~a}}1 {º}{{\textsuperscript{o}}}1 {ç}{{\c c}}1 {Ç}{{\c C}}1
    {€}{{\euro}}1
}


%
%\lstset{ %
%	language=Java,							% choose the language of the code
%	basicstyle=\ttfamily\footnotesize,		% the size of the fonts that are used for the code
%	keywordstyle=\bfseries,					% set the keyword style
%	%numbers=left,							% where to put the line-numbers
%	numberstyle=\scriptsize,				% the size of the fonts that are used for the line-numbers
%	stepnumber=2,							% the step between two line-numbers. If it's 1 each line
%											% will be numbered
%	numbersep=5pt,							% how far the line-numbers are from the code
%	backgroundcolor=\color{white},			% choose the background color. You must add \usepackage{color}
%	showspaces=false,						% show spaces adding particular underscores
%	showstringspaces=false,					% underline spaces within strings
%	showtabs=false,							% show tabs within strings adding particular underscores
%	frame=none,								% adds a frame around the code
%	%abovecaptionskip=-.8em,
%	%belowcaptionskip=.7em,
%	tabsize=2,								% sets default tabsize to 2 spaces
%	captionpos=b,							% sets the caption-position to bottom
%	breaklines=true,						% sets automatic line breaking
%	breakatwhitespace=false,				% sets if automatic breaks should only happen at whitespace
%	title=\lstname,							% show the filename of files included with \lstinputlisting;
%											% also try caption instead of title
%	escapeinside={\%*}{*)},					% if you want to add a comment within your code
%	morekeywords={*,...}					% if you want to add more keywords to the set
%}

\usepackage{xspace}

\parindent=0pt
\parskip=2pt

\setlength{\oddsidemargin}{-1cm}
\setlength{\textwidth}{18cm}
\setlength{\headsep}{-1cm}
\setlength{\textheight}{23cm}

\def\darius{\textsf{Darius}\xspace}
\def\antlr{\texttt{AnTLR}\xspace}
\def\pe{\emph{Publicação Eletrónica}\xspace}
\def\pt{\emph{Processador de Texto}\xspace}

\def\titulo#1{\section{#1}}
\def\super#1{{\em Supervisor: #1}\\ }
\def\area#1{{\em \'{A}rea: #1}\\[0.2cm]}
\def\resumo{\underline{Resumo}:\\ }


%%%%\input{LPgeneralDefintions}

\title{Processamento de Linguagens (3º ano de Curso)\\ \textbf{Trabalho Prático 1}\\ Relatório de Desenvolvimento}
\author{José Silva\\ (A74601) \and Pedro Cunha\\ (A73958) \and Gonçalo Moreira\\ (A73591) }
\date{\today}

\begin{document}

\maketitle

\begin{abstract}
Isto é um resumo do relatório de \pe focando o contexto do trb (muito sucinto),
os objectivos concretos e os resultados atingidos.\\
Algum texto curto mas que entusiasme à leitura do relatório de \pe.
\end{abstract}

\tableofcontents


\chapter{Introdução} \label{intro}

O avanço tecnológico dos últimos anos trouxe consigo a inevitabilidade de processar cada vez mais texto.
Por parte de grande parte dos utilizadores existe a necessidade frequente de fazer mudanças ou extrair determinadas
linhas de grandes quantidades de texto onde certos padrões são bastante evidentes.
O uso de expressões regulares, que proporcionam um método eficiente, poderoso e flexível no que toca ao processamento de texto,
combinado com as ferramentas que a linguagem AWK (linguagem de programação bastante mais fácil de utilizar que as linguagens mais
convencionais) proporciona um método eficiente para solucionar as necessidades descritas acima. Das ferramentas descritas anteriormente
destacam-se funções capazes de manipular strings e a utilização de arrays associativos, estruturas de dados bastante uteis e que não são
disponibilizadas por todas as linguagens de programação.

Neste primeiro trabalho prático da unidade curricular de "Processamento de Linguagens", através dos meios descritos anteriormente,
vai desenvolvido um filtro de texto capaz de realizar o processamento de transações presentes nos extratos mensais disponibilizados
pela empresa Via Verde.


\section*{Estrutura do Relatório} \

No capítulo~\ref{intro} faz-se uma pequena introdução ao problema e às ferramentas utilizadas para a resolução deste.
Para além disso, é descrita de uma forma breve a estrutura do relatório.\\
No capítulo~\ref{ae} faz-se uma análise breve mas mais detalhada do problema escolhido pelo grupo de trabalho.\\
No capítulo~\ref{cd} é descrito de uma forma sumarizada como procedemos para solucionar as várias questões propostas pelo enunciado.

No capítulo~\ref{ct} são apresentados alguns testes e respectivos resultados para comprovar o respectivo funcionamento da solução apresentada.

Finalmente, no capítulo~\ref{concl} termina-se o relatório com uma síntese do que foi dito, as conclusões e o trabalho futuro.

\chapter{Análise e Especificação} \label{ae}
\section{Descrição informal do problema}
É fornecido um ficheiro xml, que corresponde ao extrato mensal emitido pela Via Verde para um dos seus utentes. 
Pretende-se que se desenvolva um \pt para ler esse mesmo ficheiro e retirar a informação requisitada, apresentada 
em mais detalhe a baixo, na Especificação dos Requisitos.
\section{Especificação dos Requisitos}
\subsection{Dados}
Como já foi referido, é fornecido um ficheiro xml com informação correspondente ao estrato mensal emitido 
pela Via Verde para um dos seus utentes.
Este ficheiro contém no início informação do utente e também do mês
de emissão. 
Depois destes dados, seguem-se todas as transações do cliente.
Cada uma destas transações, contém a data, as horas de entrada e saída, o
local de entrada e saída, a importância paga, o desconto, o iva, o operador
e ainda o tipo de transação(Portagens ou Parques de estacionamento).
No final do ficheiro, é apresentado o total gasto no mês de emissão e ainda
o valor que diz respeito ao iva.
\subsection{Pedidos}
O utente pode ter efetuado entrada em alguns dos dias do mês a que este extrato se refere. 
Posto isto, é solicitado que se calcule o número de entradas em cada dia do mês.
Considera-se também relevante obter a lista de locais de saída pelos quais o utente passou durante todo o mês.
Chegando ao fim do mês, uma das informações mais importantes para o utente é, muito provavelmente, o total gasto nesse mesmo mês. 
É então pedido que se obtenha o total que o utente gastou. Pretende-se também que seja calculado quanto desse total é que corresponde 
ao valor despendido apenas em parques de estacionamento.\par
Para obter a informação necessária será utilizado o Sistema de Produção GAWK, especificando os padrões de frases que se pretende encontrar com recurso a expressões regulares.

\chapter{Concepção/desenho da Resolução} \label{cd}
\section{Estruturas de Dados}
Pensando nos requisitos para este projeto, é fácil de perceber que será necessário 
guardar informação relativa a alguns tópicos. Por exemplo, será crucial guardar a informação que diz respeito 
à quantidade de entradas num determinado dia. Para guardar informação deste tipo, serão utilizados arrays associativos. 
Embora apenas seja requisitado que se liste os diferentes locais de saída, serão utilizados arrays associativos de 
forma a guardar também quantas vezes passou por cada um desses locais. Relativamente a cada dia, serão ainda guardados os 
totais gastos em cada dia num array associativo que, embora não seja um requisito, considera-se informação importante. 
Acrescentando ainda aos extras já apresentados, armazena-se informação relativa ao valor despendido em parques e também em auto-estradas.
\section{Algoritmos}
No início do programa, serão atribuidos todos os valores necessários a algumas variáveis que serão uteis para o corpo do programa.
Por exemplo strings, inteiros e formatos para a função printf.
É também muito importante a definição de um Field Seperator adequado para facilitar o processamento do ficheiro.\par
O corpo do programa será na forma de condição e ação respetiva, como é típico usando GAWK. 
Na maioria das condições, estão especificadas expressões regulares que um ou mais 'fields' da linha atual tem que satisfazer. 
Sabendo que esse 'field' satisfez a expressão regular, guarda-se informação valiosa para o resultado final do programa.\par
Chegando à fase final de execução, trata-se a informação obtida, de forma a apresentar os resultados em formato apelativo. 
Por exemplo, usando html com strings definidas no início e não só.

\chapter{Codificação e Testes} \label{ct}
\section{Alternativas, Decisões e Problemas de Implementação}
\section{Testes realizados e Resultados}
Mostram-se a seguir alguns testes feitos (valores introduzidos) e
os respectivos resultados obtidos:



\chapter{Conclusão} \label{concl}
Síntese do Documento~.\\
Estado final do projecto; Análise crítica dos resultados~.\\
Trabalho futuro.

\appendix
\chapter{Código do Programa}

Lista-se a seguir o código  do programa  que foi desenvolvido.

\lstinputlisting{procViaverde}%input de um ficheiro

\bibliographystyle{alpha}
\bibliography{relprojLayout}



\end{document}
